\chapter{Analisi e Conclusioni}
\label{cha:intro}
\vspace{5mm}

Per discutere di ciò che il refactor ha portato al progetto, sono state divise le conclusioni in due parti prendendo in considerazione prima il lato tecnologico e poi il lato umano. Ponendo che il progetto è in fase di terminazione non vi è la possibilità di accedere a dati definitivi per cui quello che riporterò di seguito potrebbe essere soggetto a variazioni future.\vspace{5mm}

Prendendo in considerazione il lato umano, è subito risultato lampante come, nonostante la divisione iniziale del lavoro, i membri del team siano stati in grado di interscambiarsi facilmente tra sviluppo lato server e lato client. La presenza di una linguaggio condiviso tra client e server ha velocizzato le operazioni di integrazione tra le varie piattaforme e le code review settimanali. Infatti il vantaggio principale di utilizzare un linguaggio comune è quello di rendere più veloce la comprensione delle feature scritte da altri sviluppatori e di trovare eventuali bug o problemi in modo più immediato.\vspace{5mm}

Grazie a questo tipo di approccio si è potuta delineare una tecnica di divisione dei compiti non più per zona di lavoro (server o client) ma per funzionalità. Si è notato che la velocità di sviluppo aumentava se chi aveva il compito di implementare una particolare funzionalità, si occupava sia del lato client che del lato server. Questo ha accelerato lo sviluppo delle Rest API, evitando di dover eseguire lo step intermedio di accordarsi sulla loro struttura. Ora chi sviluppa la nuova feature lato utente ha anche il compito di sviluppare le Api lato server, creando un prodotto meno frammentato e più integrato. \vspace{5mm}

Un altro punto focale è stata la possibilità di condividere conoscenze in modo più diretto. Molte delle tecnologie e dei pattern di programmazione utilizzati principalmente a lato client hanno contaminato il lato server e vice versa. Librerie come Redux sono agnostiche e possono adattarsi facilmente al contesto server. Pattern come la programmazione funzionale possono essere distribuiti senza difficoltà nell'intera codebase del progetto. Con uno stack full Javascript, come quello proposto, si possono risolvere senza complessità aggiuntiva problemi di \emph{Coding conventions} forzandone uno unico in tutta la code base e superando i limiti tecnologici imposti da un cambio di linguaggio tra due aree.\vspace{5mm}

\section{Confronto tra le due versioni}\vspace{5mm}  

Confrontando gli stack delle due applicazioni in termini di tecnologie utilizzate possiamo vedere i motivi per cui le scelte fatte migliorano il software e sotto quali aspetti. \vspace{5mm}

Grazie all'utilizzo del medesimo framework lato mobile e lato web è stato possibile condividere gran parte del codice e delle logiche di interazione con le api lato server. Un altra zona di condivisione è l'implementazione di Redux. Data la sua agnosticità sulla piattaforma utilizzata, questa libreria è un perfetto esempio di come l'utilizzo di uno stack basato interamente su Javascript porti ad un riutilizzo del codice in modo importante ma sopratutto in modi impraticabili precedentemente. In particolare tra l’applicativo admin sviluppato come Web Application e gli applicativi nativi si è condivisa la totalità dell’implementazione gestendo lo store allo stesso modo. Le differenze principali sono sulla libreria che permette di interfacciarsi con le api lato server. Seppur molti metodi sono condivisi, quelli relativi alla mutazione dei dati sono riservati ai soli utenti autenticati, cioè i dipendenti comunali. Tale peculiarità però non influisce nella possibilità di riutilizzare parte della libreria perchè essa implementa sia la sezione admin che la sezione utente senza privilegi.\vspace{5mm} 

Inoltre, rispetto alla versione precedente, vi è una somiglianza molto forte tra quello che è l’applicativo web e l’applicativo nativo: le due applicazioni non sono compatibili ma condividono gran parte delle logiche e dei pattern utilizzando, di fatto, lo stesso framework. Ciò permette di avere un livello di complessità all'approccio di questo progetto più bassa della versione precedente. Ciò comporta tempi più corti per l'apprendimento dei pattern e delle logiche e di conseguenza risulta in una miglior produttività per un'eventuale nuova figura da inserire nel team.\vspace{5mm}
	
L'utilizzo di un database a file con la possibilità di configurare e passare ad un database classico si è rivelata molto importante per la scalabilità dell’applicativo. SqLite risulta più comodo in fase di deploy. Di contro, non ha le prestazioni di un database canonico. La possibilità di configurare rapidamente la tecnologia da utilizzare è un plus importante. Ciò è stato possibile grazie all’impiego di un ORM come Sequelize\cite{Sequelize}. Esso permette anche di implementare velocemente migrazioni e seeder restando sempre agnostico sulla tecnologia da utilizzare a lato database. La versione precedente conteneva una pesante assunzione sull’utilizzo di un database MySql e ciò abbassava sostanzialmente la flessibilità e il riutilizzo del prodotto.\vspace{5mm}

Il lato “admin”, cioè quello che va a sostituire Filemaker nella raccolta dati, è stato sviluppato con React. La scelta di sviluppare una SPA\cite{SPA} per questo compito è stata dettata da un bisogno di rendere questa operazione iterabile e ripetibile. Nel caso in cui il cliente volesse cambiare dei testi o correggere delle traduzioni può farlo in autonomia, senza dover passare da una figura che traduca le modifiche richieste aggiungendole a database manualmente. Questa tecnica inoltre astrae ulteriormente la struttura del database durante il processo di inserimento dei dati offrendo quindi flessibilità su modifiche future ad essa.

\section{Punti critici}\vspace{5mm}

L'utilizzo di una sola tecnologia in più zone distinte dello stack porta con sè i numerosi vantaggi descritti precedentemente. Và detto però, che se nel caso trattato, le tecnologie Javascript andavano a completare in modo ottimale le necessità applicative, questo può non essere sempre applicabile. In linea generale è sempre meglio scegliere la tecnologia migliore per la funzionalità che si vuole creare. Un esempio è la capacità computazionale limitata di Javascript che a confronto con Python\cite{Python} non permette di svolgere in modo efficiente operazioni su matrici, limitandone l'uso per quanto riguarda l'ambito del machine learning. In questo caso Javascript potrebbe far eseguire le operazioni di calcolo sfruttando l'ambiente fornito da Node: potrebbe eseguire i processi e restare in attesa di risultati sfruttando l'asincronicità dell'I/O fornita dal framework. Tali processi, disegnati per eseguire micro task di computazione matematica, potrebbero essere sviluppati in c o c++ in modo da essere il più ottimizzati possibile per il compito. Tale approccio, però, offre il fianco ad una serie di problematiche legate alla manutenibilità della parte in c++ e alla dipendenza non diretta del software Javascript a tale libreria. Per avere questo tipo di dipendenza è necessario crearla attraverso una struttura software per interrompere l'esecuzione o mostrare degli errori all'avvio in caso tali dipendenze non siano esplicitate.

\vspace{5mm}La scelta di utilizzare un database a file per conservare i dati sui vari punti rende problematico scalare il prodotto. Infatti, nel caso si voglia far pilotare questo applicativo da un loadbalancer\cite{LoadBalancing} in grado di animare istanze a seconda del carico, ogni istanza non sarebbe in grado di condividere dati con le altre istanze, creando un grave problema di consistenza. Per questa ragione la scelta di utilizzare questo tipo di tecnologia è fallimentare in quest'ottica. Una soluzione per dare la possibilità di scalare all'applicativo è quella di sfruttare la flessibilità di Sequelize\cite{Sequelize} che permette di mantenere la stessa struttura dei modelli ma di cambiare la tecnologia del database. Grazie a questa tipo di astrazione software sarà possibile adattare il nuovo applicativo ad altri contesti di utilizzo.









