\chapter{Analisi e Conclusioni}
\label{cha:intro}
\vspace{5mm}

Per discutere di ciò che il refactor ha portato dividerò le conclusioni in due parti prendendo in considerazione prima il lato tecnologico e poi il lato umano. Ponendo che il progetto è in fase di terminazione non vi è la possibilità di accedere a dati definitivi per cui quello che riporterò di seguito potrebbero essere soggetto a variazioni future.

\vspace{5mm} Previa analisi è necessario fornire un contesto per qualificare al meglio i dati che andrò a descrivere, il primo fattore sono il numero di sviluppatori dedicati a questo progetto. La composizione dell'organico è il seguente, un grafico, due programmatori ed un project manager il quale compito era gestire la qualità del prodotto e far rispettare le tempistiche promesse al cliente.

\vspace{5mm}Prendendo in considerazione il lato umano è subito risultato lampante come, seppur la divisione dei compiti dello sviluppo era fatta per ambienti e cioè lato client e lato server, la facilità con cui il team ha potuto interscambiarsi è stata molto evidente. Si è mostrato come, seppur la codebase fosse nuova e solo la logica generale fosse conosciuta, possedere una 'lingua franca' in cui esprimersi è stato fondamentale per velocizzare tutte quelle operazioni che richiedevano di muoversi in un differente. Un altro esempio di questo è la velocità con cui si sono eseguite le code review settimanali. Possedere un dialetto comune permette di rendere più veloce la comprensione delle features aggiunte dal collega e trovare eventuali bug o problemi in modo più immediato. Ecco che grazie a questo tipo di approccio si è potuto delineare una tecnica di divisione dei compiti non più per zona di lavoro (server o client) ma per funzionalità. Si è notato che la velocità di sviluppo aumentava se, isolata una features, chi aveva il compito di implementarla creava sia il lato client che il lato server. Questo ha evitato anche lo sviluppo di rest api ridondanti con dati mai utilizzati a front end, aumentando l'efficenza del software, evitando di dover eseguire lo step intermedio di accordarsi, per ogni nuova funzionalità che si andava ad inserire, sulla forma delle api. Ora di fatto chi sviluppa la funzionalità lato utente ha anche il compito di sviluppare le api che dovranno essere consumate per fornire tale funzionalità creando un prodotto meno frammentato e più integrato. 

\vspace{5mm}Un altro punto focale è stata la possibilità di condividere conoscenze in modo più diretto. Molte delle tecnologie e dei pattern di programmazione utilizzati principalmente a lato client hanno contaminato il lato server e vice versa. Librerie come Redux sono agnostiche e possono adattarsi facilmente al contesto server e pattern come la programmazione funzionale possono essere distribuite senza difficoltà nell'intera codebase del progetto evitando "l'effetto palude". Tale effetto è quando durante lo sviluppo una parte di codice non mantenuta diventa ingestibile a causa dell'incapacità dei colleghi di fare refactor su quella parte perché scritta con un certo linguaggio non conosciuto da tutti i componenti del team, o con un pattern conosciuto solamente dal suo scrittore. Con uno stack full Javascript come quello proposto nella soluzione descritta in questa tesi si può risolvere senza overhead questo genere di problemi forzando un vero standard in tutta la code base superando i limiti tecnologici imposti da un cambio di linguaggio tra due aree.

\vspace{5mm}Ulteriore appunto che è emerso durante lo sviluppo è stata la possibilità di condividere alcune logiche tra client e server, tagliando notevolmente i tempi di sviluppo. Tutte le logiche per il recupero dei dati sono state racchiuse in una piccola libreria che è stata condivisa tra gli applicativi mobili e il lato admin. La scrittura di un solo pezzo di codice destinato ad uno scopo preciso e portabile su più piattaforme ha inoltre evitato bug derivati da modifiche alle api dato che per più volte è bastato aggiornare internamente la libreria senza dover cambiare null'altro.

\vspace{5mm}Un ultimo punto che voglio includere che non è legato ad una situazione generica ma solamente al mio ambiente lavorativo; essendo Farnedi ICT in primis un azienda che fa supporto tecnico ad aziende e p.a. parte dei suoi dipendenti non hanno una conoscenza diretta di sviluppo software. Javascript è risultato essere molto chiaro anche a chi è al di fuori del campo, permettendo anche al lato manageriale di entrare, in minima parte, nello sviluppo. Tale considerazione potrebbe essere legata alla mia sola realtà aziendale ma è stata di grande importanza per la produzione di funzionalità qualitativamente più in linea con le richieste del direttivo.

\vspace{5mm}Il lato economico del progetto ha visto dei miglioramenti legati sopratutto alla flessibilità del linguaggio che ha permesso di produrre codice con scope più ampio, permettendo modifiche e integrazioni in tempi più brevi. Tale flessibilità data da una tipizzazione dinamica ha permesso di sterzare più volte, con facilità, la direzione dello sviluppo del prodotto.

