\chapter{Strumenti e tecnologie}
\label{cha:intro}
\vspace{5mm}
Questo capitolo descrive le varie tecnologie Javascript disponibili sul mercato e in che modo possono essere combinate per costruire il nuovo stack applicativo.

\section{Javascript}\vspace{5mm}

Javascript è nato come linguaggio di scripting per arricchire l'interazione con le pagine web. Si tratta di un linguaggio interpretato e tipizzato dinamicamente, possiede un Garbage Collector\cite{GC} e si appoggia, per essere analizzato ed eseguito, ad un motore che solitamente risiede nel browser.

Uno dei motori più performanti è quello utilizzato nel web browser Chrome ed è conosciuto con il nome di V8\cite{V8}. La particolarità di tale engine è la sua natura Open Source che ha permesso negli anni di essere utilizzato in altri contesti al di fuori del browser. Questo tipo di diversificazione nei possibili ambienti di impiego di Javascript ha portato alla nascita di numerosi framework e tool anche lato server.

\subsection{Frameworks Javascript}\vspace{5mm}

	L’ecosistema Javascript per il web è molto ricco di soluzioni per tutti gli ambiti di impiego. Per quanto riguarda il lato applicativo front-end vi sono numerose scelte, sia che si intenda operare nel browser che fuori da esso. I principali framework per lo sviluppo web front-end sono React\cite{React}, AngularJs\cite{angular}, Vue	\cite{vue} e Ember\cite{ember}. I primi tre implementano il pattern MVC “Model-View-Controller” che implica la separazione di ciò che detiene i dati da ciò che li renderizza all’utente attraverso una logica che fa da moderatore del flusso. Sia Angular che Vue che Ember offrono nativamente \emph{“two-way data binding”} ovvero la capacità di mantenere la sincronia tra View e Model. React, d’altro canto, è stato pensato per essere una libreria e non come un vero e proprio framework, infatti il suo approccio è puramente legato alla costruzione dell’interfaccia grafica, slegata da quelle che sono tutte le implementazioni possibili nella gestione del flusso dei dati. \vspace{5mm}

Nel caso di applicativi enterprise che devono gestire molti dati, questa parte può risultare complessa e particolarmente sensibile a bug. Per questo, soluzioni come Redux possono risultare funzionali per risolvere questo genere di problemi. Lo scopo di tale libreria è quello di contenere lo stato applicativo e fare in modo che sia sempre in uno stato certo e consistente, permettendo la modifica di esso solo attraverso delle “Action”. Ad ogni azione corrisponde una tipologia che il reducer interpreta per modificare lo stato. Il reducer è una funzione pura che prende in input lo stato attuale e l’azione eseguita e ritorna il nuovo stato applicativo. Essendo i reducer funzioni pure, sono prevedibili, prive di “Side effects” e facilmente testabili.\vspace{5mm} 

Per quanto riguarda la parte applicativa per le due piattaforme mobili, le scelte possibili sono racchiuse in Cordova/Phonegap, Ionic e React-Native. I primi due sono dei “wrap” attorno ad una webView nativa con una serie di API rese disponibili a Javascript che gira all’interno del browser che a sua volta risiede nella webView. Tale soluzione, seppur funzionale, non è ottimale per il caso d’uso specifico di Open Air Museum. Infatti l’applicativo mobile necessita di avere accesso al bluetooth e ad altre API non disponibili negli ambienti di Cordova/PhoneGap. Da notare che Ionic, seppur basato sulla medesima tecnologia, mette a disposizione delle API per interagire in modo completo con il bluetooth. Questo e la sua maturità lo rendono un'ottima scelta per sviluppare la parte mobile dell’applicativo.\vspace{5mm} 

D’altro canto, React-Native è un framework basato sull’utilizzo di un motore di compilazione che non fa altro che prendere i costrutti Javascript, dichiarati utilizzando librerie, e convertirli in codice nativo a seconda della piattaforma. Inoltre permette, se necessario, di interfacciarsi con l’ambiente nativo a piacimento, dando la possibilità di utilizzare un costrutto chiamato “bridge” che offre un’interfaccia a Javascript verso il nativo e viceversa. La differenza principale tra i sistemi che usano Cordova e React-native è che il primo ha un ideale legato al concetto di \emph{“Write once, run anywhere”} mentre il secondo a quello di \emph{“Learn once, use everywhere”}.\vspace{5mm}

Anche per la parte server è disponibile un'ampia scelta di tecnologie, che seppur orientate alla stessa soluzione, risultano molto diverse tra di loro. Vi sono librerie come ExpressJs che sono pensate per essere molto leggere e di lieve impatto nella strutturazione dei dati. D'altra parte, ve ne sono altre, come MeteorJs e SailsJs, che offrono un profondo controllo su tutti gli aspetti dell’applicativo e su una serie di features: la generazione automatica di API seguendo la specifica delle “CROUD operation”, la generazione automatica di modelli per descrivere nuove entità e la gestione automatica dell’autenticazione e degli utenti.

\section{Sviluppo mobile}\vspace{5mm}

Al giorno d'oggi sono disponibili sul mercato una vasta gamma di smartphone che variano per dimensione e potenza. Tutti condividono la possibilità di utilizzare App: software provenienti da terze parti, installabili mediante uno store digitale messo a disposizione dal produttore del sistema operativo.\vspace{5mm}

Solitamente per lo sviluppo di App si utilizzano i tool messi a disposizione dall'azienda produttrice del sistema operativo e cioè, nel caso di iOS e Android: rispettivamente XCode e Android Studio. Ogni OS ha a disposizione un sdk che permette di sfruttare l'hardware del dispositivo: le SDK di sistemi operativi differenti, come si può immaginare, presentano molte differenze, sia tecniche che logiche. La differenza principale sta nel linguaggio utilizzato: iOS utilizza ObjectiveC o Swift mentre Android si avvale di Java o Kotlin. Nel caso si desideri sviluppare per entrambi i sistemi operativi, l'enorme distanza tra i due comporta la necessità di avere una conoscenza approfondita di entrambi i linguaggi e dei rispettivi sdk.\vspace{5mm}

\subsection{Sviluppo mobile e Javascript}\vspace{5mm}

In alternativa alle modalità di sviluppo descritte in precedenza, vi sono molteplici soluzioni che offrono la possibilità di sviluppare per entrambi i dispositivi, iOS ed Android, utilizzando Javascript. Si differenziano per due aspetti fondamentali:
\begin{itemize}
\item Soluzioni Ibride
\item Soluzioni Compilate
\end{itemize}

Con un approccio ibrido, l'applicativo consiste in una Web Application che viene eseguita all'interno di un browser che ha accesso all'hardware dello smartphone mediante un' interfaccia software. Questa interfaccia può essere vista come un layer al di sopra degli sdk nativi che permettono alla web app di accedere a funzionalità come il bluetooth e la fotocamera.\vspace{5mm}

La soluzione compilata invece, utilizza uno stratagemma diverso. Tutti i componenti descritti attraverso una particolare struttura vengono compilati dal framework e convertiti in componenti nativi, mentre la logica applicativa viene lanciata su di un thread separato che comunica con la controparte nativa attraverso un costrutto software chiamato \emph{bridge}. Questo thread utilizza un un motore per eseguire il codice Javascript, che viene fornito, in modalità sviluppo, da un server locale che ha il compito di compilare il sorgente; mentre viene incluso nei binari post compilazione una volta che l'applicativo deve essere rilasciato sullo store. Tale approccio permette di avere prestazioni più elevate e maggior flessibilità rispetto all'ibrido, ma risulta più complesso. L'assenza di un' layer che astrae i rispettivi sdk obbliga quindi lo sviluppatore ad utilizzare direttamente il reame nativo sfruttando la comunicazione data dal bridge.\vspace{5mm}

La scelta sulla tipologia da utilizzare va fatta a seconda delle specifiche progettuali. Se non sono necessari particolari accessi all'hardware del dispositivo o non sono necessarie prestazioni molto elevate, l'approccio ibrido risulta più efficiente e rapido. La soluzione compilata è da preferire in contesti dove sono necessari particolari accessi all'hardware o è necessario l'utilizzo di una libreria disponibile solamente nel lato nativo.


\section{Audioguide}\vspace{5mm}

Vi sono disponibili diversi prodotti che offrono la possibilità di creare audioguide per determinate zone o città. Alcuni di questi utilizzano un approccio di \emph{selezione luogo}: chi utilizza l’applicativo deve prima indicare a quale zona è interessato per poi essere reindirizzato alla sezione relativa di quell’area geografica. Altri prodotti come Open Air Museum utilizzano un approccio \emph{mono-scopo}: l’audioguida è relativa ad una sola area specifica e sviluppata ad Hoc per quel caso d'uso. \vspace{5mm}

Il primo approccio descritto permette di dover mantenere un solo prodotto e di servire molteplici enti, questo però limita molto la customizzazione che tali applicativi possono possedere. Al contrario, il secondo approccio permette una customizzazione elevata, essendo il prodotto sviluppato appositamente per l’ente. Questo risulta dispendioso se si vogliono servire un gran numero di realtà, infatti ognuno degli applicativi richiederà probabilmente funzionalità e caratteristiche differenti.\vspace{5mm}

Un ulteriore punto di riflessione riguarda le funzionalità: la totalità dei competitor presi in considerazione permette di visualizzare contenuti multimediali collegati a punti di interesse o percorsi, ma in pochi offrono un servizio di localizzazione attivo il cui scopo è guidare l'utente all’interno della zona di interesse. Gli applicativi che presentano questa features utilizzano come tecnologia di localizzazione il gps che per zone all’aperto, come lo possono essere città o parchi naturali, si rivela essere un'ottima soluzione. Di contro, per quanto riguarda la localizzazione indoor, come in un museo, questa tecnologia risulta imprecisa e poco utilizzabile. IBeacon in questo caso risulta la tecnologia migliore per localizzazione degli utenti dato che non deve fare affidamento su triangolazioni satellitari per calcolare la posizione dell'utente.\vspace{5mm}

\section{Tecnologie utilizzate in Open Air Museum 1.0}\vspace{5mm}

\subsection{Filemaker}\vspace{3mm}

Filemaker è un database con cui è possibile creare applicativi personalizzati. Open Air Museum utilizza questa tecnologia per creare l’interfaccia per l'inserimento di punti e percorsi che viene utilizzata da alcuni addetti comunali. Essi forniranno i media, le descrizioni e i testi tradotti per le varie lingue supportate. Tale applicativo è standalone e quindi slegato dal lato server che espone le API. Quindi l’importazione dei dati provenienti da questa soluzione deve essere fatta manualmente. Ciò comporta uno svantaggio non indifferente e ne preclude una scalabilità rapida del prodotto. La motivazione dietro alla scelta di questa tecnologia era quella di avere il più rapidamente possibile un prodotto che mettesse il comune nelle condizioni di fornire dati consistenti.\vspace{5mm}

\subsection{Node Js}\vspace{5mm}

	Come descritto in precedenza, Nodejs è un runtime di Javascript al di fuori del browser, utilizza Chrome V8\cite{V8} come motore per eseguire Javascript e fornisce una suite di interfacce per la comunicazione con l'hardware. L'implementazione dell'interazione con l'hardware avviene attraverso un I/O non bloccante\cite{AsincIO}. Queste API mettono a disposizione la possibilità di utilizzare la scheda di rete, e quindi di avviare server TCP e HTTP, e di dare accesso al file system. Il tutto attraverso un'interfaccia lato Javascript asincrona. \vspace{5mm}
	
	In Open Air Museum questo framework è utilizzato per costruire il lato server ed esporre una serie di API Http Rest che permettono la fruizione dei dati. Tutti i media, quindi audio e immagini, vengono serviti da questa piattaforma che inoltre gestisce l’autenticazione e la gestione degli utenti.\vspace{5mm}
	
	\subsection{MySql e SqLite}\vspace{5mm}
	
	Per la gestione dei database è stato utilizzato MySql per la parte server e SqLite per la parte mobile. Sono entrambe soluzioni relazionali ma con una sostanziale differenza: la prima utilizza un applicativo che gestisce le interazioni con il database, accettando richieste e inviando dati attraverso un'interfaccia tcp; la seconda invece utilizza una gestione a file in cui non sono necessari applicativi intermedi ma l'estrazione dei dati è fatta attraverso il file system. E' chiaro come la seconda soluzione sia più adatta su dispositivi che dispongono di poca potenza e una batteria limitata dato che non vi è necessità di avviare un ulteriore processo che gestisca l'interazione con i dati.\vspace{5mm}
	
	\subsection{Java per Android e Swift}\vspace{5mm}
	
	Per scrivere gli applicativi mobile è stato utilizzato Java su Android Studio e Swift su XCode, rispettivamente per l'applicativo Android e l'applicativo IOs. L'utilizzo dei linguaggi nativi ha permesso un maggior accesso a quelle che sono le funzionalità dell'hardware ed in particolare al bluetooth. Grazie alla tecnologia iBeacon è possibile localizzare gli utenti e mostrare contenuti coerenti con la loro posizione. Tale tecnologia è supportata da un gran numero di dispositivi che montano bluetooth LE, e posseggono una versione del sistema operativo superiore al 4.3 per Android e 7.0 per IOs.
	
\section{Tecnologie utilizzate in Open Air Museum 2.0}\vspace{5mm}

	\subsection{React}\vspace{5mm}

React\cite{React} è una Libreria Javascript per la costruzione di interfacce grafiche a componenti e lo sviluppo di \emph{single page application} (SPA). Tale libreria è basata sul concetto di \emph{Dom virtuale} e \emph{One-way data flow}. Il primo è un astrazione del Dom HTML che permette di eseguire manipolazioni dello stesso in maniera molto più efficiente, e che viene poi renderizzato a seconda delle modifiche effettuate. Il secondo, invece, indica la metodologia utilizzata da React per trasferire i dati attraverso la virtualizzazione appena descritta. L'efficienza di React deriva dalla scelta di gestire il passaggio dei dati in un solo verso, e cioè da nodo padre a nodo figlio.\vspace{5mm}

React utilizza una sintassi chiamata JSX per descrivere i componenti che andranno poi renderizzati mediante l'interazione tra Dom virtuale e Dom HTML. Tale sintassi ricorda l'html ma si tratta di zucchero sintattico. Infatti è necessario configurare un Transpiler come Babel per poter utilizzare questo tipo di sintassi. Una volta che il codice in JSX viene trasposto, può essere utilizzato da qualunque browser. Infatti si tratta di semplice codice Javascript che utilizza i metodi forniti da React per interagire con il DOM Virtuale. Ciò permette di costruire la pagina html nella struttura virtuale che poi React renderizzerà in modo efficiente nel DOM del browser.


	\subsection{Redux}\vspace{5mm}
	
	Redux è una libreria Javascript che permette di gestire lo stato applicativo come un unico oggetto statico, modificabile soltanto attraverso della azioni ben definite. Ogni modifica viene fatta attraverso una funzione pura\cite{PureFunction} chiamata reducer che trasforma a seconda dell'implementazione una parte specifica dello store. La firma della funzione prende due parametri: il primo è l'oggetto che rappresenta lo stato attuale dell'applicativo, il secondo indica l'azione che si sta compiendo. L'azione dovrà sempre avere una tipologia, mentre il payload di dati è facoltativo. Ciò che ritorna il reducer corrisponde a un oggetto che diventa il nuovo stato applicativo.\vspace{5mm}
	
	Una delle specifiche di Redux è che i Reducer siano sincroni, e che quindi all'interno di essi non sia possibile eseguire richieste tramite la rete o utilizzando risorse hardware come l'accesso alla memoria. Per sopperire a questa mancanza, si utilizza un costrutto chiamato \emph{middleware}. Il compito del middleware è quello di eseguire operazioni quando una specifica azione viene inviata al reducer. Le operazioni possono essere asincrone e l'unica specifica è che alla terminazione esse dovranno lanciare un altra azione sullo store. In questo modo si può integrare l'implementazione di Redux con delle Rest API: eseguendo un'azione per avviare la richiesta ed una per inserire i dati nello store a richiesta terminata.\vspace{5mm}
	
\subsection{React-Native}\vspace{5mm}

	React-Native è l'implementazione di React per ambiente nativo. Si basa sul concetto di visualizzazione del Dom descritto in precedenza con l'unica differenza che la renderizzazione viene fatta tramite un processo che converte i vari nodi del Dom in componenti grafici nativi. In questo modo si mantiene la resa del codice nativo ma esso viene pilotato attraverso il motore di React. Tale approccio permette di scrivere applicativi nativi utilizzando le conoscenze che si posseggono per l'ambiente web senza per forza rinunciare alle prestazioni dovute alle costrizioni imposte da un web browser. Le API di React-native permettono inoltre di controllare del codice nativo, mediante una specifica dichiarazione. La comunicazione tra il reame Javascript e il reame nativo avviene attraverso un costrutto software chiamato bridge. Tale interfaccia permette di avere una comunicazione full duplex tra il thread Javascript ed altri thread nativi. Tale comunicazione avviene in modo completamente asincrono e non bloccante.\vspace{5mm}

\subsection{Express}\vspace{5mm}

	Express è l'implementazione del pattern middleware per Node Js. Tale framework permette di gestire con granularità il routing delle API e dà la possibilità di centralizzare la gestione degli errori. L'implementazione di tale pattern si riflette in un applicativo che esegue le computazioni attraverso una cascata di funzioni chiamate una dopo l'altra in un certo ordine. Ognuna di esse ha la facoltà di decidere se continuare la catena o interrompere la computazione. Questo permette di avere una struttura affidabile per gestire il routing delle Rest API rendendo molto veloce e predittivo lo sviluppo.\vspace{5mm}

\subsection{SqLite}\vspace{5mm}

	SqLite è un database \emph{“file-based”} utilizzato lato server per il salvataggio delle informazioni fornite dagli admin attraverso l’applicativo web per la raccolta dei dati. Tale scelta è legata alla flessibilità di questa tecnologia, che permette di gestire internamente all’applicativo la creazione del database, e non attraverso una dipendenza esterna. Questo è stato fatto per semplificare ulteriormente il deploy eliminando le necessità di aggiungere una configurazione ulteriore all’ambiente.\vspace{5mm}


