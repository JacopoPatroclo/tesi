\chapter{Strumenti e tecnologie}
\label{cha:intro}
\vspace{5mm}
Questo capitolo descrive le varie tecnologie Javascript disponibili sul mercato e in che modo possono essere combinate per costruire il nuovo stack applicativo.

\section{Javascript}\vspace{5mm}

Javascript è nato come linguaggio di scripting lato client 

[ Descrivo come funziona Javascript e V8 ]

\subsection{Frameworks Javascript}\vspace{5mm}

	L’ecosistema Javascript per il web è molto ricco di soluzioni per tutti gli ambiti di impiego. Per quanto riguarda il lato applicativo front-end vi sono numerose scelte, sia che si intenda operare nel browser che fuori da esso. I principali framework per lo sviluppo web front-end sono React, AngularJs, Vue e Ember. I primi tre implementano il pattern MVC “Model-View-Controller” che implica la separazione di ciò che detiene i dati da ciò che li renderizza all’utente attraverso una logica che fa da moderatore del flusso. Sia Angular che Vue che Ember offrono nativamente \emph{“two-way data binding”} ovvero la capacità di mantenere la sincronia tra View e Model. React, d’altro canto, è stato pensato per essere una libreria e non come un vero e proprio framework, infatti il suo approccio è puramente legato alla costruzione dell’interfaccia grafica, slegata da quelle che sono tutte le implementazioni possibili nella gestione del flusso dei dati. \vspace{5mm}

Nel caso di applicativi enterprise che devono gestire molti dati, questa parte può risultare complessa e particolarmente sensibile a bug. Per questo, soluzioni come Redux possono risultare funzionali per risolvere questo genere di problemi. Lo scopo di tale libreria è quello di contenere lo stato applicativo e fare in modo che sia sempre in uno stato certo e consistente, permettendo la modifica di esso solo attraverso delle “Action”. Ad ogni azione corrisponde una tipologia che il reducer interpreta per modificare lo stato. Il reducer è una funzione pura che prende in input lo stato attuale e l’azione eseguita e ritorna il nuovo stato applicativo. Essendo i reducer funzioni pure, sono prevedibili, prive di “Side effects” e facilmente testabili.\vspace{5mm} 

Per quanto riguarda la parte applicativa per le due piattaforme mobili, le scelte possibili sono racchiuse in Cordova/Phonegap, Ionic e React-Native. I primi due sono dei “wrap” attorno ad una webView nativa con una serie di API rese disponibili a Javascript che gira all’interno del browser che a sua volta risiede nella webView. Tale soluzione, seppur funzionale, non è ottimale per il caso d’uso specifico di Open Air Museum. Infatti l’applicativo mobile necessita di avere accesso al bluetooth e ad altre API non disponibili negli ambienti di Cordova/PhoneGap. Da notare che Ionic, seppur basato sulla medesima tecnologia, mette a disposizione delle API per interagire in modo completo con il bluetooth. Questo e la sua maturità lo rendono un'ottima scelta per sviluppare la parte mobile dell’applicativo.\vspace{5mm} 

D’altro canto, React-Native è un framework basato sull’utilizzo di un motore di compilazione che non fa altro che prendere i costrutti Javascript, dichiarati utilizzando librerie, e convertirli in codice nativo a seconda della piattaforma. Inoltre permette, se necessario, di interfacciarsi con l’ambiente nativo a piacimento, dando la possibilità di utilizzare un costrutto chiamato “bridge” che offre un’interfaccia a Javascript verso il nativo e viceversa. La differenza principale tra i sistemi che usano Cordova e React-native è che il primo ha un ideale legato al concetto di \emph{“Write once, run anywhere”} mentre il secondo a quello di \emph{“Learn once, use everywhere”}.\vspace{5mm}

Anche per la parte server è disponibile un'ampia scelta di tecnologie, che seppur orientate alla stessa soluzione, risultano molto diverse tra di loro. Vi sono librerie come ExpressJs che sono pensate per essere molto leggere e di lieve impatto nella strutturazione dei dati. D'altra parte, ve ne sono altre, come MeteorJs e SailsJs, che offrono un profondo controllo su tutti gli aspetti dell’applicativo e su una serie di features: la generazione automatica di API seguendo la specifica delle “CROUD operation”, la generazione automatica di modelli per descrivere nuove entità e la gestione automatica dell’autenticazione e degli utenti.\vspace{5mm}

\section{Sviluppo mobile}\vspace{5mm}

Al giorno d'oggi sono disponibili sul mercato una vasta gamma di smartphone che variano per dimensione e potenza. Tali applicativi permettono di utilizzare software proveniente da terze parti, installabili mediante uno store digitale messo a disposizione, solitamente, dal mantenitore del sistema operativo. Tali software sono indicati con il nome di Apps o Application.\vspace{5mm}

I metodi canonici per lo sviluppo di Apps con queste tecnologie sono l'utilizzo dei tool messi a disposizione dal mantenitore del sistema operativo e cioè, XCode e Android studio. Ogni sistema operativo ha a disposizione un sdk che permette di sfruttare l'hardware del dispositivo; tale sdk però è relegato ad un solo sistema operativo, con molte differenze, sia tecniche che logiche, tra i due. Un ulteriore differenza è il linguaggio, iOS utilizza ObjectiveC o Swift mentre Android Java o Kotlin. Tali dissomiglianze portano a dover, nel caso si desideri sviluppare per entrambi i sistemi operativi, a conoscere entrambi i linguaggi e i rispettivi sdk.\vspace{5mm}

\subsection{Sviluppo mobile e Javascript}\vspace{5mm}

Ciò che Javasctipt e il web in generale ha portato allo sviluppo nativo è l'approccio del 'write one run anywhere' e cioè la possibilità di avere una codebase unica per più sistemi operativi. Javasctipt negli anni si è posto come 'lingua franca' del Web e in grado di raggiungere zone molto lontane da quella per cui era stato originariamente progettato. Il parco di framework che offrono questo tipo di approccio sono molteplici e implementano tecniche differenti per raggiungere il medesimo scopo:
\begin{itemize}
\item Soluzione Ibrida
\item Soluzione Compilata
\end{itemize}

La prima utilizza un approccio ibrido e cioè l'applicativo è di fatto una applicativo web che viene eseguito all'interno di un browser che ha però capacità speciali di interazione con l'hardware dello smartphone. Di fatto offre un layer al di sopra degli sdk nativi che permettono alla web app di accedere a funzionalità native come il bluethoot e la fotocamera.\vspace{5mm}

La soluzione compilata invece utilizza uno stratagemma diverso. Tutti i componenti descritti attraverso una particolare interfaccia vengono compilati dal framework che gli converte in componenti nativi, mentre la logica applicativa in viene lanciata su di un thread separato che comunica attraverso un costrutto software con la controparte nativa. Questo thread utilizza un un motore per eseguire il codice Javascritp, che viene fornito, in modalità svilluppo, da un server locale che ha il compito di accorpare il sorgente; mentre viene incluso nel binario compilato una volta che l'applicativo deve essere rilasciato. Tale approccio permette di avere prestazioni più elevate e maggior flessibilità rispetto all'ibrido ma risulta più complesso, non essendo presenti layer che astraggono i rispettivi sdk ma è necessario seguire le direttive imposte dal sistema operativo.


\section{Audioguide}\vspace{5mm}

Vi sono disponibili diversi prodotti che offrono la possibilità di creare audioguide per determinate zone o città. Alcuni di questi utilizzano un approccio di “selezione luogo” e cioè chi utilizza l’applicativo deve prima indicare a quale zona è interessato per poi essere reindirizzato alla sezione relativa a quell’area geografica. Altri prodotti come Open Air Museum utilizzano un approccio “mono scopo” e cioè l’audioguida è relativa ad una sola area specifica e sviluppata ad Hoc per quello scopo. L’approccio descritto per primo permette di dover mantenere un solo prodotto e di servire molteplici enti, questo però limita molto la customizzazione che tali applicativi possono possedere. Di contro il secondo approccio permette una customizzazione elevata, essendo il prodotto sviluppato appositamente per l’ente; questo però comporta un overhead non indifferente per chi mantiene tale applicativo. Infatti se si vogliono servire un gran numero di realtà risulta dispendioso dover gestire e mantenere molti applicativi diversi, ognuno con le sue peculiarità e caratteristiche.\vspace{5mm}

Un ulteriore punto di riflessione sono le funzionalità. La totalità dei competitor presi in considerazione permette di visualizzare contenuti multimediali collegati a punti di interesse o percorsi, ma in pochi offrono un servizio di localizzazione attivo a guidare all’interno della zona di interesse. La totalità degli applicativi presi in considerazione che hanno questa features utilizzano come tecnologia di localizzazione il gps che per zone all’aperto, come lo possono essere città o parchi naturali, è un ottima soluzione; però per quanto riguarda la localizzazione indoor come in un museo per esempio, risulta imprecisa e poco utilizzabile. IBeacon in questo caso risulta la tecnologia migliore per localizzazione degli utenti dato che non deve fare affidamento a triangolazioni satellitari.\vspace{5mm}

Quello che offrirà al nuova versione di Open Air Museum è un approccio misto rispetto a quello descritto all’inizio di questo capitolo, ciò che si andrà a realizzare andrà a rientrare nella categoria “mono scopo” ma grazie all’elevata flessibilità delle impostazioni di avvio sarà possibile customizzare in modo profondo l’applicativo mantenendo allo stesso tempo una documentazione esplicita sulle proprie features. Infatti il file di configurazione è in formato Json\cite{JSON} quindi facilmente accessibile sia ad umani sia a computer.\vspace{5mm}

\section{Tecnologie utilizzate in Open Air Museum}\vspace{5mm}

Indicherò le varie tecnologie utilizzate per l'attuale versione dell'applicativo.\vspace{5mm}

\subsection{Filemaker}\vspace{5mm}

Filemaker è una piattaforma per creare applicativi personalizzati votati alla categoria “gestionale” sviluppata da Apple. Open Air Museum utilizza questa tecnologia per creare l’interfaccia per la raccolta dati che verrà utilizzata da alcuni addetti comunali che appunto arricchiranno l’applicativo con i testi e i media provenienti da questo software. Tale applicativo è standalone e quindi slegato dall’applicativo lato server che espone le API e l’importazione dei dati provenienti da questa soluzione deve essere fatta manualmente. Ciò comporta uno svantaggio non indifferente che preclude una scalabilità rapida del prodotto. La scelta di questa tecnologia era quella di avere il più rapidamente possibile un prodotto in grado di iniziare a raccogliere dati il prima possibile e qualunque altra soluzione, seppur migliore in un ottica futura, avrebbe impiegato maggior tempo di sviluppo.\vspace{5mm}

\subsection{Node Js}\vspace{5mm}

	Come descritto in precedenza, Node js è un runtime di Javascript al di fuori del browser.Tale programma, come dicuterò più avanti, utilizza Chrome V8\cite{V8} come motore per eseguire Javascript ed grazie ciò NodeJs è stato fornito di una suite di interfacce per la comunicazione con l'hardware. In particolare questa particolare implementazione viene indicata come I/O non bloccante\cite{AsincIO}. La particolarità di questa soluzione è che mette a disposizione una serie di API a basso livello che permettono di utilizzare la scheda di rete e quindi di avviare server TCP e HTTP oltre che accesso al file system quindi lettura e scrittura su disco. In Open air museum questo framework è utilizzato per costruire il lato server ed esporre una serie di API Http Rest che permettono la fruizione dei dati. Tutti i media, quindi audio e immagini vengono serviti anch’essi da questa piattaforma che oltre a questo gestisce anche l’autenticazione degli utenti e la gestione degli stessi. \vspace{5mm}
	
	\subsection{MySql e SqLite}\vspace{5mm}
	
	Per la gestione dei database è stato utilizzato MySql per la parte server, mentre per la parte mobile SqLite. Sono entrambe soluzioni relazionali, la differenza tra le due è che la prima utilizza un applicativo che gestisce le interazioni con il database, accettando richieste e inviando dati attraverso un interfaccia tcp. Il secondo invece utilizza una gestione a file e cioè non sono necessari applicativi intermedi ma l'estrazione dei dati è fatta attraverso il file system. E' chiaro come la seconda soluzione sia più adatta su dispositivi che dispongono di poca potenza e una batteria limitata dato che non vi è necessità di avviare un ulteriore processo che gestisca l'interazione con i dati.
	
	\subsection{Java per Android e Swift}\vspace{5mm}
	
	Per sviluppare gli applicativi mobili si sono utilizzati Java con Android Studio e Swift con XCode, rispettivamente per l'applicativo Android e IOs. L'utilizzo dei linguaggi nativi ha permesso un maggior accesso a quello che sono le funzionalità dell'hardware ed in particolare il bluethoot. Grazie alla tecnologia IBeacon è possibile localizzare gli utenti e mostrare contenuti coerenti con la loro posizione. Tale tecnologia è supportata da un gran numero di dispositivi che montano bluethoot LE, e posseggono una versione del sistema operativo superiore al 4.3 per android e 7.0 per IOs.
	
\section{Tecnologie utilizzate in Open Air Museum 2.0}\vspace{5mm}
Indicherò le varie tecnologie utilizzate per la nuova versione dell'applicativo ponendo particolare attenzione a che vantaggi portano.\vspace{5mm}

	\subsection{React}\vspace{5mm}

React\cite{React} è una Libreria Javascript per lo sviluppo di interfacce grafiche a componenti, permette lo sviluppo di single page application (SPA). Tale libreria è basata sul concetto di Dom virtuale e One-way data flow. Il primo è un astrazione del Dom HTML che permette di eseguire manipolazioni dello stesso in maniera molto più efficiente, per poi essere renderizzato a seconda delle modifiche effettuate su di esso. Il secondo invece è il modo che utilizza React per trasferire i dati attraverso la virtualizzazione appena descritta; l'efficenza di React deriva dalla scelta di gestire il passaggio dei dati in un solo verso, e cioè da nodo padre a nodo figlio.\vspace{5mm}

React utilizza una sintassi chiamata JSX per descrivere i componenti che andranno poi renderizzati mediante l'interazione tra Dom virtuale e Dom HTML. Tale sintassi ricorda l'html ma si tratta di zucchero sintattico. Infatti è necessario configurare un Transpiler come Babel per poter utilizzare questo tipo di sintassi. Una volta che il codice in JSX viene trasposto può essere utilizzato da qualunque browser infatti si tratta di una serie di chiamate a dei metodi che react usa per creare elementi nel Dom virtuale.

	\subsection{Redux}\vspace{5mm}
	
	Redux è una libreria Javascript che permette di gestire lo stato applicativo come un unico oggetto statico, modificabile soltanto attraverso della azioni ben definite applicate ad esso. Ogni modifica viene fatta attraverso una funzione pura\cite{PureFunction} chiamata reducer che modifica a seconda dell'implementazione una parte specifica dello store. La firma di tale funzione prende due parametri, il primo è l'oggetto che rappresenta lo stato attuale dell'applicativo, il secondo è ciò che rappresenta l'azione che si sta compiendo. Tale azione dovrà sempre possedere una tipologia mentre può o non può avere un payload di dati che si riferiscono a quell'azione specifica. Ciò che ritornerà il reducer sarà un oggetto che diventerà il nuovo stato applicativo.\vspace{5mm}
	
	Una delle specifiche di Redux è quella che i Reducer siano sincroni e quindi all'interno di essi non è possibile eseguire richieste tramite la rete o utilizzando risorse hardware come l'accesso alla memoria. Per sopperire a questa mancanza si utilizza un costrutto chiamato middleware. Il compito del middleware è quello di eseguire un operazione quando una specifica azione viene eseguita. Tale operazione può essere asincrona e l'unica specifica è che alla sua terminazione essa dovrà lanciare un altra azione sullo store. In questo modo si possono integrare con delle Rest API, eseguendo un azione per avviare la richiesta ed una per inserire i dati nello store.
	
\subsection{React-Native}\vspace{5mm}

	React-Native è l'implementazione di React ma per l’ambiente nativo. Si basa sul concetto di visualizzazione del Dom descritto in precedenza con l'unica differenza che la renderizzazione viene fatta tramite un processo che converte i vari nodi del Dom in componenti grafici nativi. In questo modo si mantiene la resa del codice nativo ma pilotata tramite il motore di React. Tale approccio permette di scrivere applicativi nativi utilizzando le conoscenze che si posseggono per l'ambiente web senza per forza rinunciare alle prestazioni dovute alle costrizioni imposte da un web browser. Le API di React-native permettono inoltre di controllare mediante una specifica dichiarazione del codice nativo. La comunicazione tra il reame Javascript e il reame nativo avviene attraverso un costrutto software chiamato bridge. Tale interfaccia permette di avere una comunicazione full duplex tra il thread Javascript ed altri Nativi. Tale comunicazione avviene in modo completamente asincrono e non bloccante.\vspace{5mm}

\subsection{Express}\vspace{5mm}

	Express è l'implementazione del pattern middleware per Node Js. Tale framework permette di gestire con granularità il routing delle API e da la possibilità di centralizzare la gestione degli errori. L'implementazione di tale pattern si riflette in un applicativo che esegue le computazioni attraverso una cascata di funzioni chiamate una dopo l'altra in ordine, ed ognuna di esse ha la facoltà di decidere se continuare la catena o interrompere la computazione. Questo permette di avere una struttura affidabile per gestire il routeing delle Rest API rendendo molto veloce e predittivo lo sviluppo.\vspace{5mm}

\subsection{SqLite}\vspace{5mm}

	SqLite è “File-based” database utilizzato lato server per il salvataggio delle informazioni fornite dagli utenti admin attraverso l’applicativo web per la raccolta dei dati. Tale scelta è stata fatta per la flessibilità di questa tecnologia che permette di gestire internamente all’applicativo la creazione dello stesso e non attraverso una dipendenza esterna. Questo è stato fatto per semplificare ulteriormente il deploy eliminando le necessità di aggiungere una configurazione ulteriore all’ambiente.\vspace{5mm}


