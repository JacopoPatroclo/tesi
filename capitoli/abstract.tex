\thispagestyle{empty}

\begin{flushleft}
	
  {\bf \Huge Abstract}

\vspace{4cm}

Fin dal primo momento in cui l’industria dello sviluppo software ha iniziato a muovere i primi passi, si è cercato di adottare linguaggi e tecnologie che permettessero il minor sforzo con il massimo risultato,[*io odio questa frase secondo me è da cambiare] rimanendo comunque flessibili e adattabili[chi? il soggetto della frase non è chiaro.]. Fu con l’avvento di internet che linguaggi \emph{“write once, run anywhere”} iniziarono a mostrare il loro appeal, infatti con una fetta sempre più crescente di utenti che si collegavano in rete aumentavano proporzionalmente gli ambienti da supportare. Tale frammentazione non è stata tuttora risolta e per lo sviluppo di applicativi che utilizzano il web è necessario conoscere molteplici tecnologie.\vspace{5mm}

\setlength{\parindent}{5ex}
Il web ha sempre rappresentato una finestra fondamentale per raggiungere una grande fetta di utenza. Utilizzarlo come vettore per catalizzare il turismo è stato da sempre il tentativo di enti territoriali e pubbliche amministrazioni. Open Air Museum è nato in quest'ottica, mettendo a disposizione di un comune della Romagna un tool interattivo in grado di accompagnare i turisti nella visita della città. La necessità del comune era quella di presentare i numerosi punti di interesse della città, collegandoli attraverso dei percorsi da svolgere a piedi. Open Air Museum è stato sviluppato da me e i miei colleghi del reparto di sviluppo software di Farnedi ICT.\vspace{5mm}

[sembra tutto interrotto a metà, troppo presto.]

\end{flushleft}