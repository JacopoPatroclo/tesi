\thispagestyle{empty}

\begin{flushleft}
	
  {\bf \Huge Abstract}

\vspace{4cm}

Ad oggi lo sviluppo software di applicativi client-server dispone di una ampia scelta di linguaggi e tecnologie. Solitamente per sviluppare con successo un applicativo è necessario integrare assieme molteplici di queste. Tale operazione non sempre è banale ed è fondamentale conoscere in modo approfondito gli ecosistemi che si desiderano integrare. Questa frammentazione rende necessario l'impiego di una serie di figure distribuite su tutto lo spettro tecnologico. Tutto ciò risulta in un incremento dei costi aziendali per medio-piccole realtà che desiderano affacciarsi a questo genere di mercato.\vspace{5mm}

La rapida diffusione dei dispositivi mobili ha portato enti territoriali e pubbliche amministrazioni a utilizzare questo canale come nuovo vettore per catalizzare il turismo: Open Air Museum è nato in quest'ottica, mettendo a disposizione di un comune della Romagna un tool interattivo in grado di accompagnare i turisti nella visita della città. Gli applicativi di questo progetto sono stati sviluppati dal reparto software di Farnedi ICT, integrando assieme quattro tecnologie diverse: Filemaker, Nodejs, iOS e Android. Tale frammentazione ha costretto l'azienda ad avvalersi di un numero molto elevato di figure, aumentando i costi di sviluppo e diminuendo, di conseguenza, il profitto. Inoltre per mantenere lo stack tecnologico di Open Air Museum sarebbero necessarie le medesime figure, rendendo i costi del progetto insostenibili.\vspace{5mm}

La mia tesi si propone di risolvere questa problematica riscrivendo la totalità degli applicativi utilizzando un solo linguaggio di programmazione. Tale operazione è possibile grazie ad una serie di tecnologie basate su Javascript che andranno a sostituire le quattro in uso in Open Air Museum.

\end{flushleft}