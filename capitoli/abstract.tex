\thispagestyle{empty}

\begin{flushleft}
	
  {\bf \Huge Abstract}

\vspace{4cm}

Fin dal primo momento in cui l’industria dello sviluppo software ha iniziato a muovere i primi passi, si è cercato di adottare linguaggi e tecnologie che permettessero il minor sforzo con il massimo risultato, rimanendo comunque flessibili e adattabili. Fu con l’avvento di internet che linguaggi “write once, run anywhere” iniziarono a mostrare il loro appeal, infatti con una fetta sempre crescente di utenti che si collegavano in rete aumentavano proporzionalmente gli ambienti da supportare. Tale frammentazione non è stata tuttora risolta e per lo sviluppo di applicativi che utilizzano il web è necessario conoscere molteplici tecnologie.\vspace{5mm}

\setlength{\parindent}{5ex}
L’obiettivo di questa tesi è dimostrare che è possibile sostituire lo stack di un prodotto precedentemente sviluppato con un'architettura canonica - che comprendeva quattro tecnologie diverse - in uno interamente basato su Javascript, ottenendo una maggiore manutenibilità e migliorando l’esperienza di sviluppo sia in termini lavorativi che in termini economici. L’utilizzo di questo linguaggio è stato dettato dal suo supporto su gran parte delle piattaforme, consumer e non, disponibili sul mercato.\vspace{5mm}

Il lavoro di questa tesi è stato svolto nell’ambito del progetto Open Air Museum, sviluppato da me in precedenze nel reparto di sviluppo software della Farnedi ICT per un comune dell’Emilia-Romagna, ed è nato dalla necessità di eseguire un refactor architetturale del prodotto.  Molta attenzione sarà data ai vantaggi e svantaggi reali che questo comporta e come ciò si sia riflesso sull’azienda per cui lavoro. Particolare rilevanza sarà data inoltre alle varie tecnologie e ai framework scelti.

\end{flushleft}