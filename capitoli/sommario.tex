\chapter*{Sommario} % senza numerazione
\label{sommario}

\addcontentsline{toc}{chapter}{Sommario} % da aggiungere comunque all'indice
\vspace{5mm}

\section{Contesto}\vspace{5mm}

Un ente pubblico come può essere un piccolo comune storico italiano necessita di una comunicazione efficiente e diretta con i propri visitatori. Open Air Museum, ha come obbiettivo quello di guidare e indirizzare i turisti del Comune in questione all’interno della loro città, storicamente e culturalmente ricca, con una guida virtuale che avrebbe potuto sostituire una persona fisica come guida turistica della zona. Data la grande varietà linguistica dei possibili utenti il prodotto è stato sviluppato multilingue.\vspace{5mm}

Il progetto Open Air Museum è stato sviluppato da me e i miei colleghi del reparto di Sviluppo Software di Farnedi ICT\cite{FICT} , azienda informatica di Cesena. Consiste di un’applicazione mobile per dispositivi iOS\cite{IOS} , di una seconda applicazione per dispositivi Android\cite{ANDROID} e di un software in Filemaker\cite{FileMaker} per il caricamento dei contenuti. In particolare io mi sono occupato dello sviluppo dell’applicazione mobile per iOS\cite{IOS} in Swift 4\cite{Swift} e dell’applicativo lato server in NodeJs.\cite{Nodejs} \vspace{5mm}

Tale applicativo è stato sviluppato durante il corso dell'anno 2017 e ha visto un solo upgrade. Durante questo periodo di mantenimento del prodotto sono venute alla luce alcune criticità derivate dalle scelte tecniche.

\section{Motivazioni}\vspace{5mm}

Le motivazioni che mi hanno portato a scegliere questo tipo di progetto per la mia tesi triennale sono due. La prima è la necessità di distruggere la credenza che Javascript\cite{JS} sia un linguaggio di secondo ordine e non sia applicabile su progetti più ampi come linguaggio primario e non solo di contorno. La seconda è il desiderio di applicarmi nella progettazione di architetture software e non soltanto progettare un' piccolo applicativo, parte di un sistema più grande.

\section{Problema e Tecniche Utilizzate}\vspace{5mm}

Questa tesi si prefigge di delineare le scelte fatte dal team di sviluppo durante il refactor di tale applicativo e di discutere le scelte tecniche prese per affrontare questo progetto. Il refactor prevalentemente consiste nel mantenere le medesime funzionalità dell'applicativo precedente ma rendendolo più gestibile e mantenibile da un team più ristretto di persone, diminuendone quindi i costi di gestione.\vspace{5mm}


La scelta che viene proposta per risolvere questo problema è quella di passare ad uno stack completamente Javascript, tale approccio dovrebbe garantire una linearità di pattern e logiche condivise in tutte le parti dell'applicativo, da frontend a backend.

\section{Obiettivi}\vspace{5mm}

	L’obiettivo di questa tesi è dimostrare che è possibile, per un prodotto completo di questo tipo, riscrivere la totalità degli applicativi che lo compongono in Javascript e che questo sia un vantaggio sia dal punto di vista dei costi di gestione che dell’effettiva manutenibilità del progetto, mantenendo le medesime funzionalità e senza alterare la qualità del prodotto. Inoltre valuterò l’efficacia di adottare uno stack Javascript soppesando i costi di sviluppo e il peso tecnologico utilizzando come metro di paragone le ore impiegate nello sviluppo dello stack “tradizionale” rispetto a quello Javascript.\vspace{5mm}

\section{Conclusioni}\vspace{5mm}

Con l'utilizzo di una serie di tecnologie è stato possibile riprodurre la totalità delle funzionalità della versione precedente in quella full Javacript. Tale traguardo è stato possibile grazie al vastissimo ecosistema Javascript che mette a disposizione tool e framework per creare applicativi su tutto lo spettro; da lato server con Nodejs a mobile con React-native. Questo tipo di approccio ha migliorato molto l'esperienza di sviluppo abbassando le barriere tecnologiche tra i vari lati applicativi che impediscono a sviluppatori frontend di approcciarsi al lato server e viceversa. Con questa soluzione è stato possibile rivalutare il concetto di riutilizzo del codice, ora è possibile condividere librerie in tutto lo spettro applicativo velocizzando ulteriormente la produzione software\vspace{5mm}

Sono state riscontrate alcune limitazioni di applicazione di una soluzione come questa in altri contesti applicativi e lavorativi. Javascript è un linguaggio versatile, veloce e facile; ma non è il migliore in assoluto. Nel caso si necessiti di velocità maggiori, come nel caso del Machine Learning, è bene utilizzare altri tool e linguaggi. Nel caso di Open Air Museum lo stack sopra citato si è rivelata la soluzione adeguata. 
	
\section{Struttura della tesi}\vspace{5mm}
	
Il capitolo 2 descrive le tecnologie disponibili attualmente sul mercato candidate a sostituire quelle impiegate nella versione precedente dell’applicativo. Il capitolo 3 analizza le scelte tecnologiche fatte ponendo un confronto tra le due versioni. Il capitolo 4 definisce più nel dettaglio il porting degli applicativi mobile illustrando i processi e le criticità legate a questa operazione. Il capitolo 5 descrive la possibilità di costruire un process manager per istanze del prodotto. Infine nel capitolo 6 viene eseguita un’analisi e una valutazione del lavoro svolto, con relative conclusioni. 


