\chapter*{Sommario} % senza numerazione
\label{sommario}

\addcontentsline{toc}{chapter}{Sommario} % da aggiungere comunque all'indice
\vspace{5mm}

\emph{contesto, motivazioni, riassumi problema, tecniche utilizzate, risultati raggiunti} \vspace{5mm}

\section{Contesto}\vspace{5mm}

Un ente pubblico come può essere un piccolo comune storico italiano necessita di una comunicazione efficiente e diretta con i propri visitatori. Nel caso di Open Air Museum, si trattava di un progetto nato per guidare e indirizzare i turisti del Comune in questione all’interno della loro città, storicamente e culturalmente ricca, con una guida virtuale che avrebbe potuto sostituire una persona fisica come guida turistica della zona. Data la grande varietà linguistica dei possibili utenti il prodotto è stato sviluppato multilingue.\vspace{5mm}

Il progetto Open Air Museum è stato sviluppato da me e i miei colleghi del reparto di Sviluppo Software di Farnedi ICT, azienda informatica di Cesena. Consiste di un’applicazione mobile per dispositivi iOS, di una seconda applicazione per dispositivi Android e di un software in Filemaker per il caricamento dei contenuti. In particolare io mi sono occupato dello sviluppo dell’applicazione mobile per iOs in Swift 3 e dell’applicativo lato server in NodeJs.\vspace{5mm}

\section{Motivazioni}\vspace{5mm}

perchè hai scelto questo tipo di tesi?

\section{Problema e Tecniche Utilizzate}\vspace{5mm}

puoi anche riunire queste due nell'ultima parte 

\section{Obiettivi}\vspace{5mm}

	L’obiettivo di questa tesi è dimostrare che è possibile, per un prodotto completo di questo tipo, riscrivere la totalità degli applicativi che lo compongono in Javascript e che questo sia un vantaggio sia dal punto di vista dei costi di gestione che dell’effettiva manutenibilità del progetto, mantenendo le medesime funzionalità e senza alterare la qualità del prodotto. Inoltre valuterò l’efficacia di adottare uno stack Javascript soppesando i costi di sviluppo e il peso tecnologico utilizzando come metro di paragone le ore impiegate nello sviluppo dello stack “tradizionale” rispetto a quello Javascript.\vspace{5mm}
	
	\section{Struttura della tesi}\vspace{5mm}
	
Il capitolo 2 descrive le tecnologie disponibili attualmente sul mercato candidate a sostituire quelle impiegate nella versione precedente dell’applicativo. Il capitolo 3 analizza le scelte tecnologiche fatte ponendo un confronto tra le due versioni. Il capitolo 4 definisce più nel dettaglio il porting degli applicativi mobile illustrando i processi e le criticità legate a questa operazione. Il capitolo 5 descrive la possibilità di costruire un process manager per istanze del prodotto. Infine nel capitolo 6 viene eseguita un’analisi e una valutazione del lavoro svolto, con relative conclusioni. 


